\documentclass[11pt]{article}
\usepackage[utf8]{inputenc}
\usepackage[slovene]{babel}

\usepackage{amsthm}
\usepackage{amsmath, amssymb, amsfonts}
\usepackage{relsize}
\usepackage{mathrsfs}
\usepackage{bbm}
\usepackage{xcolor}

\newcommand{\R}{\mathbb{R}}
\newcommand{\N}{\mathbb{N}}
\renewcommand{\P}{\mathbb{P}}
\newcommand{\E}{\mathbb{E}}
\renewcommand{\c}{\mathsf{c}}
\newcommand{\set}[1]{\{#1\}}
\newcommand{\oklepaj}[1]{\left(#1\right)}
\newcommand{\1}{\mathbbm{1}}
\newcommand{\rr}{[-\infty,\infty]}
\newcommand{\ra}{\rightarrow}
\newcommand{\5}{\vspace{0.5cm}}
\renewcommand{\t}{{t \geq 0}}
\newcommand{\vp}{(\Omega, \F, \P)}
\newcommand{\fvp}{(\Omega, \F, \P, (\F_t)_\t)}
\newcommand{\ps}{(E, \mathscr{E})}
\newcommand{\pst}{(\N_0, 2^{\N_0})}
\newcommand{\xt}{(X_t)_\t}
\newcommand{\yt}{(Y_t)_\t}
\newcommand{\nt}{(N_t)_\t}
\newcommand{\ind}{\perp\!\!\!\perp}
\newcommand{\bor}{(\R, \B(\R))}
\newcommand{\ti}{0 \leq t_1 < t_2 < \ldots < t_k}

\newcommand{\B}{\mathscr{B}}
\newcommand{\EE}{\mathscr{E}}
\newcommand{\F}{\mathscr{F}}

\theoremstyle{definition}
\newtheorem{definicija}{Definicija}[section]

\theoremstyle{definition}
\newtheorem{trditev}{Trditev}[section]

\theoremstyle{definition}
\newtheorem{izrek}{Izrek}[section]

\theoremstyle{definition}
\newtheorem{metoda}{Metoda}[section]

\newtheorem*{posledica}{Posledica}
\newtheorem*{opomba}{Opomba}
\newtheorem*{komentar}{Komentar}
\newtheorem{lema}{Lema}
\newtheorem*{dokaz}{Dokaz}
\newtheorem*{posplošitev}{Posplošitev}
\newtheorem*{dogovor}{Dogovor}
\newtheorem*{sklep}{Sklep}
\newtheorem*{oznake}{Oznake}

\title{Slučajni procesi 1 - definicije, trditve in izreki}
\author{Oskar Vavtar \\
po predavanjih profesorja Janeza Bernika}
\date{2021/22}

\begin{document}
\maketitle
\pagebreak
\tableofcontents
\pagebreak

% #################################################################################################

\begin{oznake}
~
\begin{itemize}
	\item $(\Omega, \F, \P)$ verjetnostni prostor
	\item $\Lambda \neq \emptyset$ indeksna množica
	\item $(E, \EE)$ prostor stanj
\end{itemize}

\end{oznake}
\vspace{0.5cm}

% #################################################################################################

\section{Predavanje, 14.2.}
\vspace{0.5cm}

\definicija{\textit{Slučajni proces}, parametriziran z $\Lambda$ in prostorom stanj $E$, je nabor slučajnih spremenljivk $(X_\lambda)_{\lambda \in \Lambda}$, kjer je $\Omega \rightarrow E$ $\F$-merljiva slučajna spremenljivka za vsak $\lambda \in \Lambda$.}
\5

\definicija{\textit{Trajektorija} za $\omega \in \Omega$ je funkcija $t \mapsto X_t(\omega)$, si sliko $[0, \infty) \rightarrow (\R, \B(\R))$ \scriptsize{(oz. $[0,\infty)\ra (E, \EE)$)}.}
\5

\definicija{Naj bo $(X_t)_\t$ slučajni proces na $\vp$ za vrednostmi v $\ps$. Za vsak nabor $0 \leq t_1 < t_2 < \ldots < t_k$ ima $k$-razsežni slučajni vektor $(X_{t_1}, X_{t_2}, \ldots, X_{t_k})$ skupno porazdelitev. Takim porazdelitvam pravimo \textit{končno razsežne robne porazdelitve}.
\5

\posledica{Naj bosta $\xt$ in $\yt$ dva različna procesa. Če velja
$$(X_{t_1}, X_{t_2},\ldots, X_{t_k}) ~\overset{(d)}{=}~ (Y_{t_1}, Y_{t_2}, \ldots, Y_{t_k}),$$
$\forall 0 \leq t_1 < t_2 < \ldots < t_k$, $\forall k \geq 1$, potem za $\forall A \in \F_{\infty}$ in $\forall \tilde{A} \in \tilde{\F}_\infty$ \scriptsize{($\tilde{\F}_\infty:= \sigma\set{\yt}$)}}, ki sta določeni ``na enak način'', potem
$$\P(A) ~=~ \P(\tilde{A}).$$
Rečemo $\xt \overset{(d)}{=} \yt$. Če za enak nabor $t_i$ velja
$$(X_{t_1}, X_{t_2},\ldots, X_{t_k}) ~\ind~ (Y_{t_1}, Y_{t_2}, \ldots, Y_{t_k}),$$
potem $\forall A \in \F_\infty$, $\forall B \in \tilde{\F}_\infty$ velja
$$\P(A \cap B) ~=~ \P(A)\P(B),$$
procesa sta \textit{neodvisna}.}
\5

\definicija{Naj bo $\ps = \bor$, $\ti$:
$$X_{t_1} - X_{t_0}, ~X_{t_2} - X_{t_1}, ~\ldots~, ~X_{t_k} - X_{t_{k-1}}$$
so \textit{prirastki}.}
\5

\definicija{Proces $\xt$ ima \textit{neodvisne prirastke}, če so za vsak nabor $\ti$ slučajne spremenljivke/vektorji $X_{t_1} - X_{t_0}, X_{t_2} - X_{t_1}, \ldots, X_{t_k} - X_{t_{k-1}}$ neodvisni. Prirastki prek neprekrivajočih intervalov takega procesa so neodvisni.}
\5

\definicija{Proces $\xt$ ima \textit{stacionarne prirastke}, če za vsak $\ti$,  velja
$$(X_{t_1} - X_{t_0}, \ldots, X_{t_k} - X_{t_{k-1}}) ~\overset{(d)}{=}~ (X_{t_1 + h} - X_{t_0 + h}, \ldots, X_{t_k + h} - X_{t_{k-1} + h}), \quad \forall h \geq 0.$$}
\5

\definicija{\textit{L\'evyjevi procesi} so slučajni procesi z neodvisnimi, stacionarnimi ter c\`adl\`ag trajektorijami.}
\5

\definicija{~\begin{itemize}
	\item $\F_t := \sigma\set{X_s \mid 0 \leq s \leq t}$
	\item $\fvp$ filtriran verjetnostni prostor
	\item $T: \Omega \ra [0,\infty]$ je čas ustavljanja, če $\set{T \leq t} \in \F_t$, $\forall \t$
	\item $\F_T := \set{A \in \F \mid A \cap \set{T \leq t} \in \F_t, \forall t \geq 0}$ $\sigma$-algebra zgodovine časa $T$
\end{itemize}
Če je proces c\`adl\`ag, je $T$ merljiva glede na $\F_T$ in $X_T$ na $\set{T < \infty}$ merljiva glede na $\F_T$.

\definicija{\textit{Proces štetja} $\nt$ je slučajni proces s prostorom stanj $\pst \subseteq \bor$, pri katerem so trajektorije $t \mapsto N_t(\omega)$, $t \geq 0$ nepadajoče, zvezne z desne in z vrednostmi v $\N_0$. \scriptsize{Iz same definicije sledi, da je proces c\`adl\`ag}.}
\5

\definicija{\textit{Zaporedni časi skokov}:
\begin{itemize}
	\item $S_1 := \inf\set{t \mid N_t \neq N_0}$ 
	\item $S_{n+1} := \inf\set{t > S_n \mid N_t \neq N_{S_n}}\1_{\set{S_n \infty}} + \infty\1_{\set{S_n = \infty}}$
\end{itemize}
Časi skokov so časi ustavljanja.}
\5

\definicija{Definiramo \textit{višino $n$-tega skoka} $\Delta(S_n) := N_{S_n} - N_{S_n^-}$, kjer je $N_{S_n^-} := \lim_{s \nearrow S_n} N_s$. Proces štetja je \textit{enostaven}, če je $\Delta(S_n) = 1$, kadarkoli je definirana. V takem primeru velja
$$N_t ~=~ N_0 + \sum_{n \geq 1} \1_{\set{S_n \leq t}}.$$}
\5


% #################################################################################################


\end{document}