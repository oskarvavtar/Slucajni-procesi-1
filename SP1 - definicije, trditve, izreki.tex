\documentclass[11pt]{article}
\usepackage[utf8]{inputenc}
\usepackage[slovene]{babel}

\usepackage{amsthm}
\usepackage{amsmath, amssymb, amsfonts}
\usepackage{relsize}
\usepackage{mathrsfs}
\usepackage{bbm}
\usepackage{xcolor}
\usepackage[scr=boondox]{mathalfa}
\usepackage{yhmath}

\newcommand{\R}{\mathbb{R}}
\newcommand{\N}{\mathbb{N}}
\newcommand{\Z}{\mathbb{Z}}
\renewcommand{\P}{\mathbb{P}}
\newcommand{\E}{\mathbb{E}}
\renewcommand{\c}{\mathsf{c}}
\newcommand{\set}[1]{\{#1\}}
\newcommand{\oklepaj}[1]{\left(#1\right)}
\newcommand{\1}{\mathbbm{1}}
\newcommand{\rr}{[-\infty,\infty]}
\newcommand{\ra}{\rightarrow}
\newcommand{\5}{\vspace{0.5cm}}
\renewcommand{\t}{{t \geq 0}}
\newcommand{\vp}{(\Omega, \F, \P)}
\newcommand{\fvp}{(\Omega, \F, \P, (\F_t)_\t)}
\newcommand{\ps}{(E, \mathscr{E})}
\newcommand{\pst}{(\N_0, 2^{\N_0})}
\newcommand{\ind}{\perp\!\!\!\perp}
\newcommand{\bor}{(\R, \B(\R))}
\newcommand{\ti}{0 \leq t_1 < t_2 < \ldots < t_k}

\newcommand{\xt}{(X_t)_\t}
\newcommand{\yt}{(Y_t)_\t}
\newcommand{\nt}{(N_t)_\t}
\newcommand{\tns}{(\tilde{N}_s)_{s \geq 0}}
\newcommand{\tnt}{(\tilde{N}_t)_{t \geq 0}}
\newcommand{\xn}{(X_n)_{n \geq 0}}
\newcommand{\yn}{(Y_n)_{n \geq 0}}
\newcommand{\ft}{(\F_t)_\t}

\newcommand{\ber}{\text{Ber}}
\newcommand{\hpp}{\text{HPP}}
\newcommand{\poi}{\text{Pois}}
\renewcommand{\exp}{\text{Exp}}
\newcommand{\mult}{\text{Mult}}
\newcommand{\geom}{\text{Geom}}

\newcommand{\B}{\mathscr{B}}
\newcommand{\EE}{\mathscr{E}}
\newcommand{\F}{\mathscr{F}}
\newcommand{\U}{\mathcal{U}}
\renewcommand{\o}{\mathscr{o}}
\renewcommand{\L}{\mathscr{L}}
\newcommand{\M}{\hat{M}}

\theoremstyle{definition}
\newtheorem{definicija}{Definicija}[section]

\theoremstyle{definition}
\newtheorem{trditev}{Trditev}[section]

\theoremstyle{definition}
\newtheorem{izrek}{Izrek}[section]

\theoremstyle{definition}
\newtheorem{metoda}{Metoda}[section]

\newtheorem*{posledica}{Posledica}
\newtheorem*{opomba}{Opomba}
\newtheorem*{komentar}{Komentar}
\newtheorem{lema}{Lema}
\newtheorem*{dokaz}{Dokaz}
\newtheorem*{posplošitev}{Posplošitev}
\newtheorem*{dogovor}{Dogovor}
\newtheorem*{sklep}{Sklep}
\newtheorem*{oznake}{Oznake}

\title{Slučajni procesi 1 - definicije, trditve in izreki}
\author{Oskar Vavtar \\
po predavanjih profesorja Janeza Bernika}
\date{2021/22}

\begin{document}
\maketitle
\pagebreak
\tableofcontents
\pagebreak

% #################################################################################################

\begin{oznake}
~
\begin{itemize}
	\item $(\Omega, \F, \P)$ verjetnostni prostor
	\item $\Lambda \neq \emptyset$ indeksna množica
	\item $(E, \EE)$ prostor stanj
\end{itemize}

\end{oznake}
\vspace{0.5cm}

% #################################################################################################

\section{Uvod v procese štetja}
\vspace{0.5cm}

\definicija{\textit{Slučajni proces}, parametriziran z $\Lambda$ in prostorom stanj $E$, je nabor slučajnih spremenljivk $(X_\lambda)_{\lambda \in \Lambda}$, kjer je $\Omega \rightarrow E$ $\F$-merljiva slučajna spremenljivka za vsak $\lambda \in \Lambda$.}
\5

\definicija{\textit{Trajektorija} za $\omega \in \Omega$ je funkcija $t \mapsto X_t(\omega)$, si sliko $[0, \infty) \rightarrow (\R, \B(\R))$ \scriptsize{(oz. $[0,\infty)\ra (E, \EE)$)}.}
\5

\definicija{Naj bo $(X_t)_\t$ slučajni proces na $\vp$ za vrednostmi v $\ps$. Za vsak nabor $0 \leq t_1 < t_2 < \ldots < t_k$ ima $k$-razsežni slučajni vektor $(X_{t_1}, X_{t_2}, \ldots, X_{t_k})$ skupno porazdelitev. Takim porazdelitvam pravimo \textit{končno razsežne robne porazdelitve}.
\5

\posledica{Naj bosta $\xt$ in $\yt$ dva različna procesa. Če velja
$$(X_{t_1}, X_{t_2},\ldots, X_{t_k}) ~\overset{(d)}{=}~ (Y_{t_1}, Y_{t_2}, \ldots, Y_{t_k}),$$
$\forall 0 \leq t_1 < t_2 < \ldots < t_k$, $\forall k \geq 1$, potem za $\forall A \in \F_{\infty}$ in $\forall \tilde{A} \in \tilde{\F}_\infty$ \scriptsize{($\tilde{\F}_\infty:= \sigma\set{\yt}$)}}, ki sta določeni ``na enak način'', potem
$$\P\set{A} ~=~ \P\set{\tilde{A}}.$$
Rečemo $\xt \overset{(d)}{=} \yt$. Če za enak nabor $t_i$ velja
$$(X_{t_1}, X_{t_2},\ldots, X_{t_k}) ~\ind~ (Y_{t_1}, Y_{t_2}, \ldots, Y_{t_k}),$$
potem $\forall A \in \F_\infty$, $\forall B \in \tilde{\F}_\infty$ velja
$$\P\set{A \cap B} ~=~ \P\set{A}\P\set{B},$$
procesa sta \textit{neodvisna}.}
\5

\definicija{Naj bo $\ps = \bor$, $\ti$:
$$X_{t_1} - X_{t_0}, ~X_{t_2} - X_{t_1}, ~\ldots~, ~X_{t_k} - X_{t_{k-1}}$$
so \textit{prirastki}.}
\5

\definicija{Proces $\xt$ ima \textit{neodvisne prirastke}, če so za vsak nabor $\ti$ slučajne spremenljivke/vektorji $X_{t_1} - X_{t_0}, X_{t_2} - X_{t_1}, \ldots, X_{t_k} - X_{t_{k-1}}$ neodvisni. Prirastki prek neprekrivajočih intervalov takega procesa so neodvisni.}
\5

\definicija{Proces $\xt$ ima \textit{stacionarne prirastke}, če za vsak $\ti$,  velja
$$(X_{t_1} - X_{t_0}, \ldots, X_{t_k} - X_{t_{k-1}}) ~\overset{(d)}{=}~ (X_{t_1 + h} - X_{t_0 + h}, \ldots, X_{t_k + h} - X_{t_{k-1} + h}), \quad \forall h \geq 0.$$}
\5

\definicija{\textit{L\'evyjevi procesi} so slučajni procesi z neodvisnimi, stacionarnimi ter c\`adl\`ag trajektorijami.}
\5

\definicija{~\begin{itemize}
	\item $\F_t := \sigma\set{X_s \mid 0 \leq s \leq t}$
	\item $\fvp$ filtriran verjetnostni prostor
	\item $T: \Omega \ra [0,\infty]$ je čas ustavljanja, če $\set{T \leq t} \in \F_t$, $\forall \t$
	\item $\F_T := \set{A \in \F \mid A \cap \set{T \leq t} \in \F_t, \forall t \geq 0}$ $\sigma$-algebra zgodovine časa $T$
\end{itemize}
Če je proces c\`adl\`ag, je $T$ merljiva glede na $\F_T$ in $X_T$ na $\set{T < \infty}$ merljiva glede na $\F_T$.

\definicija{\textit{Proces štetja} $\nt$ je slučajni proces s prostorom stanj $\pst \subseteq \bor$, pri katerem so trajektorije $t \mapsto N_t(\omega)$, $t \geq 0$ nepadajoče, zvezne z desne in z vrednostmi v $\N_0$. \scriptsize{Iz same definicije sledi, da je proces c\`adl\`ag}.}
\5

\definicija{\textit{Zaporedni časi skokov}:
\begin{itemize}
	\item $S_1 := \inf\set{t \mid N_t \neq N_0}$ 
	\item $S_{n+1} := \inf\set{t > S_n \mid N_t \neq N_{S_n}}\1_{\set{S_n \infty}} + \infty\1_{\set{S_n = \infty}}$
\end{itemize}
Časi skokov so časi ustavljanja.}
\5

\definicija{Definiramo \textit{višino $n$-tega skoka} $\Delta(S_n) := N_{S_n} - N_{S_n^-}$, kjer je $N_{S_n^-} := \lim_{s \nearrow S_n} N_s$. Proces štetja je \textit{enostaven}, če je $\Delta(S_n) = 1$, kadarkoli je definirana. V takem primeru velja
$$N_t ~=~ N_0 + \sum_{n \geq 1} \1_{\set{S_n \leq t}}.$$}
\5

\pagebreak

% #################################################################################################

\section{Homogeni Poissonov proces  \\
(Poissonov tok)}
\5

\definicija{Naj bo $\lambda>0$ dano realno število. \textit{Enostavni proces štetja} $\nt$, za katerega je $N_0=0$, je $\hpp(\lambda)$, če zanj velja ena od naslednjih trditev:
\begin{enumerate}

\item Proces $\t$ ima neodvisne in stacionarne prirastke in za $\forall \t$ je
$$N_t ~\overset{(d)}{=}~ \poi(\lambda t).$$
Če to velja, potem:
\begin{itemize}
	\item $N_{t+s} - N_t \overset{(d)}{=} N_s - \underbrace{N_0}_{=~0} = N_s \overset{(d)}{=} \poi(\lambda s)$
	\item za ``porazdelitev proces'' potrebujemo (načeloma) pokazati
	\begin{align*}
	(N_{t_0}, N_{t_1},\ldots,N_{t_k}) ~&\overset{(d)}{=}~ (N_{t_0}, N_{t_0} + (N_{t_1} - N_{t_0}),\ldots,N_{t_0} + (N_{t_1} - N_{t_0}) + (N_{t_k} - N_{t_{k-1}}) \\
	&=~ \varphi(N_{t_0}, N_{t_1} - N_{t_0}, \ldots, N_{t_k} - N_{t_{k-1}})
	\end{align*}
\end{itemize}

\item \textit{Infinetzimalna karakterizacija}: proces $\nt$ ima neodvisne in stacionarne prirastke in velja
\begin{align*}
\P\set{N_t = 1} ~&=~ \lambda t + \o(t) \\
\P\set{N_t \geq 2} ~&=~ \o(t),
\end{align*}
kjer je $\o(t) \in \set{g(t) \mid \lim_{t \searrow 0} \frac{g(t)}{t} = 0}$.

\item \textit{Lastnost medprihodnih časov}: zaporedni časi skokov $S_n$ so končni s.g., t.j. $\P\set{S_n < \infty} = 1 ~\forall n$. $S_0 = 0$. Potem je zaporedje \textit{medprihodnih časov}
$$T_i ~:=~ S_i - S_{i-1}, ~i \geq 1,$$
dobro definirano in porazdeljeno kot zaporedje n.e.p. $\exp(\lambda)$ slučajnih spremenljivk. Ta pogoj nam da tudi eksistenco $\hpp(\lambda)$. $\vp$, $(E_i)_{i \geq 1}$ n.e.p. s $\exp(\lambda)$, $W_n := \sum_{i=1}^n E_i$. Potem je
$$N_t ~:=~ \sum_{k \geq 1} \1_{\set{W_k \leq t}}$$
$\hpp(\lambda)$.

\item \textit{Lastnost vrstilnih statistik}: za $\forall\t$ je $N_t \overset{(d)}{=} \poi(\lambda t)$ in pogojno na $\set{N_t = k}, k \geq 1$, je vektor
$$(S_1,\ldots,S_n) \mid \set{N_t = k} ~\overset{(d)}{=}~ (U_{(1)},\ldots,U_{(k)}),$$
kjer je $(U_{(1)},\ldots,U_{(k)})$ vektor vrstilnih statistik za vektor $(U_1,\ldots,U_k)$ z n.e.p. $\U([0,t])$ porazdeljenimi komponentami.

\end{enumerate}}
\5

\trditev[Zakon redkih dogodkov; osnovna verzija]{Naj bo $\yn$ zaporedje slučajnih $Y_n \overset{(d)}{=} \ber(n,p_n)$ in obstaja $\lim_{n \rightarrow \infty} n\, p_n = c > 0$. Potem
$$Y_n ~\xrightarrow{(d)}~ \poi(c).$$}
\5

\lema[Lema Slutskega]{Naj bosta $\xn$ in $\yn$ taka, da \hbox{$X_n \xrightarrow{(d)} X$} in $Y_n \xrightarrow{\P} c \in \R$. Potem velja 
$$X_n + Y_n \xrightarrow{(d)} X + c.$$}

\lema{Dana so števila $\lambda_1,\ldots,\lambda_n>0$, $\lambda := \sum_{i=1}^n \lambda_i$, $p_i := \frac{\lambda_i}{\lambda}$. Za nabor $Z_1,\ldots,Z_n$, slučajne spremenljivke z vrednostmi v $\N_0$ sta ekvivalentni trditvi:
\begin{enumerate}
	\item $Z_1,\ldots,Z_n$ so neodvisne in $Z_i \overset{(d)}{=} \poi(\lambda_i)$
	\item $Z := Z_1 + \ldots + Z_n \overset{(d)}{=} \poi(\lambda)$ in, pogojno na $\set{Z=k}$, $k \geq 1$, je slučajni vektor $(Z_1,\ldots,Z_n)$ porazdeljen z $\mult(k; p_1,\ldots,p_n)$, tj.
	$$\P\set{Z_1 = j_1,\ldots,Z_n = j_n \mid Z=k} ~=~ \frac{k!}{j_1! \cdot \ldots \cdot j_n!}p_1^{j_1}\cdot\ldots\cdot p_n^{j_n},$$
	kjer $j_1 + \ldots + j_n = k$.
\end{enumerate}}
\5

\opomba{V literaturi bomo pogosto zasledili naslednjo definicijo $\hpp(\lambda)$:
$$(\triangle) \begin{cases}
N_0 = 0 ~ \text{s.g.} \\
\text{prirastki neodvisni, stacionarni in}~N_t \overset{(d)}{=} \poi(\lambda t), ~\forall t \geq 0
\end{cases}$$
Ta definicija želi $N_0 = 0$ le do s.g.-enakosti in ne zahteva, da je proces štetja enostaven. Da se zlahka videti, da je proces, ki zadošča $(\triangle)$, skoraj gotovo enostaven.}
\5

\trditev[Enostavna lastnost Markova za HPP]{Naj bo $\nt$ $\hpp(\lambda)$. Za $t>0$ definiramo
$$\tilde{N}_s ~:=~ N_{t+s} - N_t, \quad s \geq 0.$$
$\tns$ je tudi $\hpp(\lambda)$ in neodvisna od $\F_t = \sigma\oklepaj{\set{N_u \mid 0 \leq u \leq t}}$.}
\5

\posledica{~\begin{itemize}
	\item $S_2 \mid \set{N_t=1}\overset{(d)}{=} t + \exp(\lambda)$
	\item $S_{N_t+1} \overset{(d)}{=} t + \exp(\lambda)$ 
	\item $\E[S_{N_t+1}] = t + \frac{1}{\lambda}$
\end{itemize}
\5

\trditev[Krepka lastnost Markova za HPP]{Naj bo $\nt$ $\hpp(\lambda)$, $\ft$ naravna filtracija in $T$ čas ustavljanja za $\ft$. Na $\set{T < \infty}$ definiramo 
$$\tilde{N}_s ~:=~ N_{T+s} - N_T.$$ 
Potem je $\tns$, pogojno na $\set{T < \infty}$, $\hpp(\lambda)$ in neodvisen od
$$\F_T ~=~ \set{A \in \F_\infty \mid A \cap \set{T \leq t} \in \F_t, ~\forall \t}.$$
\5

\posledica{Na dogodku $\set{S_1<\infty}$ je $(N_{S_1+t}-N_{S_1})_\t$ $\hpp(\lambda)$ in neodvisen od $\F_{S_1}$.}
\5

\definicija[Starost in presežek]{Za $\t$ definiramo
\begin{itemize}
	\item presežek (excess): $E_t = S_{N_t+1}-t$
	\item starost (age): $A_t = t - S_{N_t}$, ki je merljiva glede na $\F_t$
\end{itemize}
\5

\trditev[Enostavna lastnost Markova za $(E_t)_\t$ in $(A_t)_\t$]{$E_t \overset{(d)}{=} \exp(\lambda)$ in neodvisen od $\F_t$ (torej tudi od $A_t$). Za $U,V$ neodvisni $\exp(\lambda)$-porazdeljeni slučajni spremenljivki je $(E_t, A_t) \overset{(d)}{=} (U, V \wedge t)$. Če definiramo $L_t$, dolžino medprihodnega časa, ki zaobjema čas $t$, velja
\begin{align*}
L_t ~&=~ S_{N_t+1} - S_t \\
&=~ A_t + E_t
\end{align*} 
Za $L_t$ velja \textit{paradoks medprihodnega časa}:
$$\lim_{t \ra \infty} \E[L_t] ~=~ \frac{2}{\lambda},$$
medtem ko je upanje dolžin drugih medprihodnih časov $\frac{1}{\lambda}$.}
\5

\subsection*{Markiranje $\hpp(\lambda)$}
\5

\definicija[Markiranje HPP]{Naj bo $\nt$ $\hpp(\lambda)$, $(S_i)_{i \geq 1}$ zaporedje prihodov ter $(X_n)_{n \geq 1}$ zaporedje n.e.p. ($d$-razsežnih) slučajnih vektorjev, neodvisno od $\nt$. Zaporedju $(S_i, X_i)_{i \geq 1}$ pravimo \textit{markiranje} $\hpp(\lambda)$ z zaporedjem oznak (markacij) $(X_n)_{n \geq 1}$. Za $i \in \N$ je $(S_i, X_i) \in [0,\infty) \times \R^d$.}
\5

\oznake{~
\begin{itemize}

\item $\mu$ ... skupni zakon $(X_n)_{n \geq 1}$ na $\R^d$
\begin{itemize}
	\item $\mu$ je mera na $(\R^d, \B(\R^d))$
	\item $X: \Omega \ra \R^d$
	\item $B \in \B(\R^d)$: $\mu(B) = \P\set{X \in B}$
\end{itemize}
\item $\nu$ ... produktna mera na $\oklepaj{[0,\infty)\times\R^d, \B\oklepaj{[0,\infty)\times\R^d}}$, kjer je 1. faktor $\lambda$-večkratnik Lebesgueove mere na $\oklepaj{[0,\infty), \B\oklepaj{[0,\infty)}}$, 2. faktor pa $\mu$ na $(\R^d, \B(\R^d))$:
$$\nu(ds, dx) ~=~ \lambda\1_{\set{s>0}}ds \otimes d\mu(x),$$
tj. za produktno množico oblike $A = [a,b]\times C$, $0 \leq a \leq b$, $C \in \B(\R^d)$,
$$\nu(A) ~=~ \lambda(b-a) \cdot \mu(C).$$
\end{itemize}}
\5

\izrek{Naj bodo $A_1,\ldots,A_m \subset [0,\infty) \times \R^d$ Borelovo merljive paroma disjunktne množice, kjer je $\forall i$ $A_i$ omejena v časovni komponentni, tj. $\exists T > 0$ tak, da $A_i \subset [0,T]\times\R^d$, $i \in \set{1,\ldots,m}$. Postavimo
$$N(A) ~=~ \sum_{i \geq 1} \1_{\set{(S_i,X_i)\in A}}.$$
Pri teh predpostavkah so komponente $\oklepaj{N(A_1),\ldots,N(A_m)}$ neodvisne, pri čemer je
$$N(A_i) ~\overset{(d)}{=}~ \poi(\nu(A_i)).$$}
\5

\trditev[Redčenje $\hpp(\lambda)$]{Naj bo $\nt$ $\hpp(\lambda)$ in $(X_i)_{i \geq 1}$ zaporedje $\ber(p)$-porazdeljenih slučajnih spremenljivk, neodvisnih od $\nt$. Postavimo
\begin{align*}
N_t^1 ~&=~ \sum_{i \geq 1} \1_{\set{S_i \leq t, X_i=1}}, \\
N_t^0 ~&=~ \sum_{i \geq 1} \1_{\set{S_i \leq t, X_i=0}}.
\end{align*}
Očitno,
$$N_t^0 + N_t^1 ~=~ N_t, \quad t \geq 0.$$
Procesa $(N_t^0)_{t \geq 0}$, $(N_t^1)_{t \geq 0}$ sta neodvisna HPP, prvi z intenzivnostjo $\lambda_0 = (1-p)\lambda$, drugi pa z $\lambda_1 = p\lambda$. Tukaj je $\mu = p\delta_{\set{1}} + (1-p)\delta_{\set{0}}$.}
\5

\trditev[Superpozicija HPP]{Naj bo 
$$X_i ~:=~ \1_{\set{S_i = S_k^1,~\text{za nek}~k}} ~=~ \sum_{k \geq 1} \1_{\set{S_i = S_k^1}}.$$
$X_i$ je Bernoullijeva slučajna spremenljivka, ki pokaže 1, če je bil $i$-ti prihod združenega procesa $(N_t)_{t \geq 0}$ prihod, ki je prišel oz. iz $(N_t^1)_{t \geq 0}$. $(X_i)_{i \geq 1}$ ke zaporedje n.e.p. slučajnih spremenljivk, porazdeljenih s $\ber\oklepaj{\frac{\lambda_1}{\lambda_0 + \lambda_1}}$. Še več, to zaporedje je neodvisno od $\nt$.}

\pagebreak

% #################################################################################################

\section{Nehomogeni Poissonov proces}
\5

\definicija[Sprememba ure]{Naj bo $R: [0,\infty)\ra [0,\infty)$ nepadajoča funkcija, c\`adl\`ag (zaradi enostavnosti lahko predpostavimo $\R(x) = 0$, $x<0$) in $\tnt$ homogen Poissonov proces na $\vp$ z intenzivnostjo $1$. Definirajmo
$$N_t ~=~ \tilde{N}_{R(t)}, \quad t \geq 0.$$
Zlahka se prepričamo, da velja:
\begin{itemize}
	\item proces $\nt$ ima neodvisne prirastke,
	\item za $0 \leq s < t$ je $N_t - N_s \overset{(d)}{=} \poi\oklepaj{R(t)-R(s)}$.
\end{itemize}
Procesu $\nt$ rečemo nehomogen Poissonov proces z intenzivnostno mero $\mu$, kjer je $\mu$ $\sigma$-končna mera na $\oklepaj{\R^+, \B(\R^+)}$, ki jo porodi $R$ s predpisom
$$\mu\oklepaj{[0,a]} ~=~ R(a), \quad a \geq 0.$$}
\5

\definicija[Nehomogen Poissonov proces]{Naj bo $\mu$ mera na $\oklepaj{\R^+, \B(\R^+)}$, kjer $\mu\oklepaj{[0,a]}<\infty$ za vse $a \geq 0$. Definirajmo $R(t) = \mu\oklepaj{[0,t]}$, $t\geq 0$, ki je nepadajoča funkcija, zvezna z desne. Procesu štetja $\nt$, ki ima neodvisne prirastke in za katerega velja, da je za poljubno $0\leq s<t$
$$N_t - N_s ~\overset{(d)}{=}~ \poi\oklepaj{\mu\oklepaj{(s,t]}} ~=~ \poi\oklepaj{R(t)-R(s)}$$
rečemo \textit{nehomogeni Poissonov proces} z intenzivnostno mero $\mu$.}
\5

\definicija[Trenutna intenzivnost]{Naj bo $\mu$ absolutno zvezna glede na Lebesgueovo mero, $\mu \ll \L$. Po Radon-Nikodymovem izreku, $\frac{d\mu(s)}{d\L(s)} = \rho(s)$, kjer je $\rho \in L_{\text{loc}}^1\oklepaj{\R^+,\L}$ lokalno integrabilna\footnote{absolutno integrabilna na vsakem končnem intervalu}. Očitno je $\rho \geq 0$ s.p. V tem primeru je
$$R(t) ~=~ \int_0^t \rho(s)\,ds$$
in za $0 \leq u < v$ je
$$N_v - N_u ~\overset{(d)}{=}~ \poi\oklepaj{\int_u^v \rho(s)\,ds}.$$
V posebnem primeru, ko je $\rho(t) = \lambda>0$ in $R(t) = \lambda t$ na privede nazaj do $\hpp(\lambda)$.}
\5

\trditev[Nehomogena lastnost Markova]{Naj bo $\F_t := \sigma\oklepaj{\set{N_s \mid 0 \leq s \leq t}}$, $\t$, naravna filtracija za $\nt$ z intenzivnostjo $\mu$. Proces $\oklepaj{N_{t+s}-N_t}_{s\geq 0}$ je neodvisen od $\F_t$ in je nehomogen Poissonov proces, z intezivnostno mero, porojeno z 
$$R(s) ~=~ R(t+s) - R(t) ~=~ \mu\oklepaj{(0,s]}, \quad s \geq 0.$$ 
Če je $\frac{dR}{dt} = \rho(t)$ trenutna intenzivnost za $\nt$, je za $\oklepaj{N_{t+s}-N_t}_{s\geq 0}$ trenutna intenzivnost $\rho_1(s) = \rho(t+s)$, $s\geq 0$.}
\5

\trditev[Krepka nehomogena lastnost Markova]{Naj bo $T$ čas ustavljanja glede na naravno filtracijo $\ft$, je na dogodku $\set{T<\infty}$ proces $\oklepaj{N_{T+s}-N_T}_{s\geq 0}$, pogojno na $\set{T<\infty}$ in $\ft$, nehomogen Poissonov proces z intenzivnostno funkcijo,
\begin{align*}
R_1(s) ~&=~ R(T+s), \\
\rho_1(s) ~&=~ \rho(T+s), \quad s \geq 0,
\end{align*}
če $\rho$ obstaja.}
\5

\trditev{Lastnost vrstilnih  statistik]{Naj bo $\nt$ nehomogen Poissonov proces, $R(t)$, $0\leq s \leq t$ taka, da je $R(t)-R(s)>0$. Potem je 
$$\oklepaj{S_{N_s+1},\ldots,S_{N_s+k}} \mid \set{N_t-N_s=k} ~\overset{(d)}{=}~ \oklepaj{Y_{(1)},\ldots,Y_{(k)}}.$$
V posebnem, če obstaja trenutna intenzivnost $\rho$, imajo $Y_1,\ldots,Y_k$ gostoto
$$\frac{\rho(u)}{R(t)-R(s)}\1_{(s,t]}(u).$$}
\5 
\pagebreak
\komentar{~\begin{itemize}
\item Pogojna porazdelitev za $\oklepaj{S_{N_s+1},\ldots,S_{N_s+k}}$ in $\oklepaj{Y_{(1)},\ldots,Y_{(k)}}$ se ujemata na družini
$$\mathcal{U} ~=~ \set{(s_1,t_1]\times\ldots\times(s_k,t_k] \mid s<s_1\leq t_1 < \ldots <s_k\leq t_k \leq t}.$$
$\mathcal{U}$ je zaprta za neskončne preseke in generira Borelovo $\sigma$-algebro
$$\mathcal{V} ~=~ \set{(x_1,\ldots,x_k) \mid s<x_1\leq\ldots\leq x_k \leq t}.$$
Če je $\mathcal{U}$ $\pi$-sistem, po Dynkinovem $\pi$-$\lambda$ izreku sledi
$$\oklepaj{S_{N_s+1},\ldots,S_{N_s+k}} ~\overset{(d)}{=}~ \oklepaj{Y_{(1)},\ldots,Y_{(k)}}.$$ 
\item $\mu$ je Radonova mera na $\oklepaj{[0,\infty),\B\oklepaj{[0,\infty)}}$ $\iff$ $R: [0,\infty) \ra [0,\infty)$ je nepadajoča in zvezna z desne.
\end{itemize}}
\5

\trditev[Infinetzimalna karakterizacija\footnote{za primer, ko je $\rho$ zvezna}]{Imamo proces štetja $\nt$, ki ima neodvisne prirastke, skoke velikosti 1 in zvezno nenegativno funkcijo $\rho: [0,\infty) \ra [0,\infty)$,
\begin{itemize}
	\item $\P\set{N_{t+h}-N_t=1} = \rho(t)h + \o_t^1(h)$,
	\item $\P\set{N_{t+h}-N_t \geq 2} = \o_t^2(h)$,
\end{itemize}
pri čemer je $\lim_{h \searrow 0} \frac{\o_t^i(h)}{h} = 0$, $i \in \set{1,2}$, enakomerno za $t \in [0,T]$, $\forall T \geq 0$. Potem je za $\forall s<t$
$$N_t - N_s ~\overset{(d)}{=} \poi\oklepaj{R(t)-R(s)} ~=~ \poi\oklepaj{\int_s^t \rho(u)\,du}.$$}
\5

\izrek{Naj bo $\nt$ nehomogen Poissonov proces na $\vp$ z intenzivnostno mero $\mu$. Denimo, da je $R(t) = \mu\oklepaj{[0,t]}$ strogo naraščajoča zvezna funkcija in $R(0)=0$. Potem na morda razširjenem\footnote{v zapiskih piše na istem???} verjetnostnem prostoru $(\tilde{\Omega}, \tilde{\F}, \tilde{\P})$ obstaja homogen Poissonov proces $\tnt$ z intenzivnostjo $1$, da velja
$$N_t ~=~ \tilde{N}_{R(t)}, \quad \t.$$}
\5

\pagebreak

% #################################################################################################

\section{Prenovitveni procesi}
\5

\definicija[Prenovitveni proces]{Naj bo $(T_i)_{t \geq 1}$ zaporedje neodvisnih in enako porazdeljenih slučajnih spremenljivk, $T_i \sim F$. Predpostavke so:
\begin{itemize}
	\item $F(0) = \P\set{T_i=0} < 1$ ($T_i$ niso s.g. enake 0),
	\item $\mu = \E[T_i] = \int_{[0,\infty)} x\,dF(x) \leq \infty$ označuje pričakovan medprihodni čas, pri čemer dopuščamo $\mu = \infty$. Velja pa tudi $\mu > 0$.
\end{itemize}
Definiramo zaporedne čase prihodov oz. prenovitvene trenutno
\begin{align*}
S_0 ~&=~ 0, \\
S_k ~&=~ \sum_{i=1}^k T_i, \quad k \geq 1.
\end{align*}
Potem je \textit{prenovitveni proces} $\nt$ definiran z
$$N_t ~=~ \sum_{k\geq 1} \1_{\set{S_k \leq t}} ~=~ \sup\set{k \mid S_k \leq t}.$$}
\5

\komentar{Za porazdelitev $N_0$ velja
\begin{align*}
\P\set{N_0 = k} ~&=~ \P\set{S_k = 0, S_{k+1}>0} \\
&=~ \P\set{T_1 = 0,\ldots,T_k = 0, T_{k+1}>0} \\
&=~ F(0)^k(1-F(0))
\end{align*}
Torej $N_0 \overset{(d)}{=} \geom\oklepaj{1-F(0)}-1$ in $\P\set{N_0 = 0} = 1$.}
\5

\trditev[Lastnosti]{~\begin{itemize}
	\item Proces $\nt$ je nepadajoč in zvezen z desne. Verjetnost eksplozije je enaka $0$.
	\item $\P\set{N_t = \infty} = \P\set{S_k \leq t, \forall k \geq 1} = 0$ za vsak $\t$.
	\item $S_k \leq t$ le za končno mnogo $k$ s.g.
	\item Na $\set{t \geq S_k}$ je $N_t \geq k$.
	\item Ker je s.g. $S_k < \infty$ za vsak $k$, je torej $\lim_{t \rightarrow \infty} N_t = \infty$ s.g.
\end{itemize}
Vidimo, da na dogodku z verjetnosto $1$ proces $\nt$ zadošča zahtevam procesa štetja.}
\5

\definicija{Slučajn spremenljivka $T$ je \textit{aritmetična}, če $\exists a > 0$ tak, da
$$\P\set{T \in \Z a} ~=~ 1,$$
torej $T$ s.g. pokaže le večkratnike števila $a$.}
\5

\definicija[Prenovitvena mera]{Na bo $\nt$ prenovitveni proces. Za $\t$ definiramo \textit{prenovitveno mero}
$$M(t) ~=~ \E[N_t].$$
Za $t<0$ lahko postavimo $M(t) = 0$.}
\5

\definicija[$k$-ta konvolucija]{Za porazdelitveno funkcijo $F$, kjer $T_i \sim F$, definiramo $k$-to konvolucijo kot
$$F^{k*}(x) ~=~ \P\set{S_k \leq x} ~=~ \P\set{T_1 + \ldots + T_k \leq x}.$$
V posebnem je $F^{0*} = \1_{[0,\infty)}(x)$ Heavisideova funkcija, tore je porazdelitev s.g. enaka konstanti $0$. Ker je $S_{k+1} = S_k + T_{k+1}$ velja rekurzivna zveza
$$F^{(k+1)*}(x) ~=~ \int_\R F^{k*}(x-s)\,dF(s) ~=~ F^{k*} * F.$$}
\5

\komentar{Izkaže se, da za neodvisni slučajni spremenljivki $U\sim F$ in $V\sim G$ velja
$$U+V ~\sim~ F*G,$$
kjer 
$$F*G(t) ~=~ \begin{cases}
0\,; ~&t<0, \\
\int_{[0,t]} F(t-s)\,dG(s) = \int_{[0,t]}G(t-s)\,dF(s)\,; ~& \t. 
\end{cases}$$}
\5

\trditev{Vsaka nepadajoča, zvezna iz desne funkcija $H: \R \rightarrow \R$ porodi mero oz. zakon $\mu$ na $\bor$ s predpisom
$$\mu\oklepaj{(a,b]} ~=~ H(b) - H(a) ~=~ \int_{(a,b]}dH(x),$$
ki ga lahko enolično razširimo na celotno Borelovo $\sigma$-algebro.}
\5

\trditev[Lastnosti]{Za prenovitveni proces $\nt$ z medprihodno porazdelitvijo $F$ velja naslednje.
\begin{itemize}
	\item Za vsak $\t$ in $r \geq 0$ je $\E[N_t^r]<\infty$. V posebnem je $M(t)$ dobro definirana.
	\item Velja 
	$$M(t) ~=~ \sum_{k=1}^\infty F^{k*}(t).$$
	Pri tem je $M(t)$ nepadajoča in zvezna z desne. Velja še \hbox{$\lim_{t\ra\infty} M(t) = \infty$.}
	\item Za Laplace-Stieltjesevo transformacijo $\M(s)$ prenovitvene mere $M$ velja
	$$\M(s) ~=~ \int_{[0,\infty)} e^{-sx}\,dM(x) ~=~ \frac{\hat{F}(s)}{1-\hat{F}(s)}, \quad s \geq 0,$$
	pri čemer je $\hat{F}(s) = \int_{[0,\infty)} e^{-sx}\,dF(x) = \E[e^{-sT}]$ Laplace-Stieltjesova transformacija porazdelitvene funkcije $F$ oz. Laplaceova transformacija slučajne spremenljivke $T$.
\end{itemize}}
\5

\izrek[Elementarni prenovitveni izrek]{Za prenovitveno mero $M$ velja
$$\lim_{t\ra\infty} \frac{M(t)}{t} ~=~ \frac{1}{\mu} ~=~ \frac{1}{\E[T]} \in [0,\infty).$$}
\5
\pagebreak
\trditev[Asimptotske lastnosti]{~\begin{itemize}
	\item KZVŠ:
	$$\lim_{t\ra\infty} \frac{N_t}{t} = \frac{1}{\mu} \quad \text{s.g.}$$
	\item CLI: pod pogojem $\sigma^2 = \text{Var}[T]<\infty$ velja
	$$\frac{N_t - \frac{t}{\mu}}{\sigma\sqrt{\frac{t}{\mu^3}}} ~\xrightarrow{(d)}~ \mathcal{N}(0,1).$$
\end{itemize}}
\5

\definicija[Prenovitev z zaostankom]{Naj bo zaporedje medprihodnih časov $(T_i)_\t$ zaporedje neodvisnih nenegativnih slučajnih spremenljivk, pri čemer
\begin{itemize}
	\item $T_1 \sim G$,
	\item $T_2,T_3,\ldots \sim F$, 
\end{itemize}
kjer $\mu = \E[T_2] = \int_{[0,\infty)} x\,dF(x)$, $F(0) = \P\set{T_2 = 0} <1$. Kot prej naj bo (za $\t$) 
\begin{align*}
S_k ~&=~ \sum_{i=1}^k T_i, \\
N_t ~&=~ \sum_{k \geq 1} \1_{\set{S_k \leq t}}, \\
M(t) ~&=~ \E[N_t].
\end{align*}}
\5

\trditev[Lastnosti]{~\begin{itemize}
	\item Proces $\nt$ ima s.g. nepadajoče, z desne zvezne trajektorije, nima eksplozije in $\lim_{t\ra\infty} N_t = \infty$.
	\item $\E[N_t^r]<\infty$, $\forall \set{t,r}\subset\R_0^+$. V posebnem je $M(t)$ dobro definirana, z desne zvezna funkcija.
	\item Velja
	$$M(t) ~=~ \sum_{k\geq 0} G*F^{k*}(t) ~=~ \sum_{k\geq 1} G*F^{(k-1)*}(t),$$
	kjer je
	$$G*F^{k*}(t) ~=~ \int_{[0,t]} G(t-s)\,dF^{k*}(s), \quad k \geq 0, t\geq 0.$$
	\item $\hat{M}(s) = \frac{\hat{G}(s)}{1-\hat{F}(s)}$
	\item $\lim_{t\ra\infty} \frac{M(t)}{t} = \frac{1}{\mu} = \frac{1}{\E[T_2]}$
	\item $\lim_{t\ra\infty} \frac{N_t}{t} = \frac{1}{\mu}$ s.g.
\end{itemize}
Zadnji dve lastnosti pomenita, da na dolgi rok vpliv drugačne porazdelitve $T_1$ izzveni.}
\5

\trditev{Naj bo $\nt$ prenovitveni proces z zaostankom (ali brez). Za $\t$ definiramo
$$\tilde{N}_s ~:=~ N_{t+s} - N_t, \quad s\geq 0.$$
Potem je $\tns$ prenovitveni proces z zaostankom $\tilde{T}_1 = S_{N_t+1}-t$. $\tilde{T}_1, \tilde{T}_2,\ldots$ so neodvisni od $\tilde{T}_1$, enako porazdeljene kot $T_2$.}
\5

\definicija[Porazdelitev integriranega repa]{Naj bo $\nt$ prenovitveni proces z zamikom tak, da
$$\mu ~=~ \E[T_2] ~=~ \int_{[0,\infty)} x\,dF(x) < \infty.$$ \textit{Porazdelitev integriranega repa} $G_*$ za $F$ je potem podana z
$$G_*(x) ~=~ \frac{1}{\mu} \int_0^x \oklepaj{1-F(u)}\,du\1_{[0,\infty)}(x).$$}
\5

\komentar{$G_*(0) = 0$, $\lim_{x\ra\infty} G_*(x) = 1$}
\5

\trditev{Naj bodo $T_2,T_3,\ldots \sim F$, $\mu<\infty$ in $T_1 \sim G_*$. Potem velja
$$M(t) ~=~ \E[N_t] ~=~ \frac{t}{\mu}.$$}
\5

\definicija[Stacionarnost]{Za proces štetka $\nt$ rečemo, da je stacionaren, če za vsak $\t$ velja
$$(N_{t+s}-N_t)_{s\geq 0} ~\overset{(d)}{=}~ (N_s)_{s\geq 0},$$
t.j., da statistične ne moremo ugotoviti, ali spremljamo proces od začetka, ali pa šele od nekega časa $t$ naprej.}
\5

\trditev{Za prenovitveni proces z zaostankom $\nt$ vela, da je stacionaren natanko tedaj, ko je $\mu = \E[T_2]<\infty$ in ima $T_1$ porazdelitev integriranega repa glede na $T_2$.}
\5

\izrek[Blackwellov prenovitveni izrek]{Naj bo $\nt$ prenovitveni proces z medprihodno porazdelitvijo $F$, ki ni aritmetičn, $M(t) = \E[N_t]$, $\mu = \E[T] = \int_{[0,\infty)} x\,dF(x) \leq \infty$. Potem velja
$$\lim_{t\ra\infty} \oklepaj{M(t+h)-M(t)} ~=~ \frac{h}{\mu}, \quad h \geq 0.$$}
\5

\opomba{Zgornji izrek velja tudi v primeru, ko je medprihodna porazdelitev aritmetična, a v tem primeru le za take $h$, ki so večkratniki razpona $F$.}
\5

\pagebreak

% #################################################################################################

\section{Prenovitvene enačbe}
\5

\definicija[Prenovitvena enačba]{Naj bo $h: [0,\infty) \ra \R$ merljiva, lokalno omejena funkcija in $F$ porazdelitvena funkcija nenegativne slučajne spremenljivke $T$, ki ni s.g. enaka $0$, t.j. $F(0)<1$. Iščemo funkcio $g$, ki je konstantna $0$ na $(-\infty, 0)$ in $g(t-\boldsymbol{\cdot}) \in L^1(dF)$ $\forall \t$, in za katero velja
$$g(t) ~=~ h(t) + \int_{[0,t]} g(t-s)\,dF(s), \quad \t,$$
oz. na krajše
$$g ~=~ h + g*F.$$
Taki enačbi pravimo $(h,F)$-prenovitvena enačba.}
\5

\trditev{Prenovitvena mera $M(t)$ prenovitvenega procesa z medprihodno porazdelitvijo $F$ zadošča $(F,F)$-prenovitveni enačbi
$$M ~=~ F + M*F.$$}
\5

\trditev{Prenovitvena mera $M(t)$ prenovitvenega procesa z zaostankom, kjer $T_1 \sim G$, $T_2,T_3,\ldots \sim F$, zadošča $(G,F)$-prenovitveni enačbi
$$M ~=~ G + M*F.$$}
\5

\posledica{~\begin{itemize}
	\item $M = G + M*F$ sledi
	\begin{align*}
	\hat{M} ~&=~ \hat{G} + \widehat{M*F} \\
	&=~ \hat{G} + \hat{M}\cdot\hat{F} ~\Longrightarrow~ \hat{M} = \frac{\hat{G}}{1-\hat{F}}.
	\end{align*}
\end{itemize}}
\5

\definicija[Porazdelitev starosti in presežka]{Naj bosta $A_t = t-S_{N_t}$ startost in $E_t = S_{N_t+1}-t$ presežek prenovitvenega procesa $\nt$ z medprehodno porazdelitvijo $F$ ob času $t$. Definiramo
\begin{align*}
a_x(t) ~&=~ \P\set{A_t \leq x}, \\
e_x(t) ~&=~ \P\set{E_t \leq x}.
\end{align*}
Pri fiksnem $t$ sta to točno porazdelitveni funkciji starosti in preseška, lahko pa ju gledamo tudi kot funkciji $t$ pri fiksnem $x$. V tem primeru dobimo prenovitveni enačbi
\begin{align*}
a_x(t) ~&=~ (1-F(t))\1_{\set{t\leq x}} + \int_{[0,t]} a_x(t-s)\,dF(s), \\
e_x(t) ~&=~ F(t+x) - F(t) + \int_{[0,t]} e_x(t-s)\,dF(s).
\end{align*}
Funkcija $a_x(t)$ torej reši $\oklepaj{(1-F)\1_{\set{t\leq x}}, F}$-prenovitveno enačbo, $e_x(t)$ pa $\oklepaj{F(\boldsymbol{\cdot}+x)-F(\boldsymbol{\cdot}), F}$-prenovitveno enačbo.}
\5

\izrek[Obstoj in enoličnost rešitev prenovitvenih enačb]{Naj bo $h:[0,\infty)\ra\R$ merljiva, lokalno omejena funkcija, $h(x) = 0$ na $(-\infty,0)$, in $F$ porazdelitvena funkcija nenegativne slučajne spremenljivke $T$, ki ni s.g. enaka $0$, t.j. $F(0)<1$. Potem obstaja ena sama merljiva in lokalno omejena funkcija $g$, $g(x) = 0$ na $(-\infty,0)$, ki reši $(h,F)$-prenovitveno enačbo $g = h+ g*F$ in sicer je
$$g = h + h*M$$
oz.
\begin{align*}
g(t) ~&=~ h(t) + \int_{[0,t]} h(t-s)\,dM(s) \\
&=~ h(t) \sum_{k\geq 1} \int_{[0,t]} h(t-s)\,dF^{k*}(s),
\end{align*}
kjer je $M(t) = \sum_{k\geq 1} F^{k*}(t)$ prenovitvena mera procesa $\nt$ z medprihodno porazdelitvijo $F$.}
\5
\pagebreak
\definicija[Direktni Riemannov integral]{Naj bo $h: [0,\infty)\ra[0,\infty)$ nenegativna merljiva funkcija. Pravimo, da je $h$ \textit{direktno Riemannovo integrabilna} (\textit{d.R.i.}), če zadošča naslednjima pogojema:
\begin{itemize}
	\item $\forall \Delta> 0$:
	$$\sum_{k \geq 0} \oklepaj{\sup_{t\in [k\Delta,(k+1)\Delta)} h(t)} ~<~ \infty,$$
	\item $$\lim_{\Delta\searrow 0} \Delta\sum_{k \geq 0} \oklepaj{\sup_{t\in [k\Delta,(k+1)\Delta)} h(t)} ~=~ \lim_{\Delta\searrow 0} \Delta \sum_{k \geq 0} \oklepaj{\inf_{t\in [k\Delta,(k+1)\Delta)} h(t)}.$$
\end{itemize}
Če $h$ zadošča navedenima zahtevama, potem je limita v drugi zahtevi točno vrednost direktnega Riemannovega integrala $\int_0^\infty h(u)\,du$. Funkcija $h$ poljubnega predznaka je d.R.i., če sta le-taki $h^+ = h \wedge 0$ in $h^- = (-h) \wedge 0$, pri čemer je $\int_0^\infty h(u)\,du = \int_0^\infty h^+(u)\,du + \int_0^\infty h^-(u)\,du$.}
\5

\trditev[Kriteriji za d.R.i.]{~\begin{itemize}
	\item Če je nenegativna funkcija $h \geq 0$ d.R.i., potem je $h$ omejena, zvezna s.p. na $[0,\infty)$ in $\lim_{t \rightarrow \infty} h(t) = 0$.
	\item Če je $h$ merljiva in omejena funkcija na $[0,\infty)$ za katero obstaja $T\geq 0$ tak, da je $h(t) = 0 ~\forall t \geq T$ in je $h$ zvezna s.p. na $[0,\infty)$, potem je $h$ d.R.i.
	\item Če je $h \geq 0$ nenaraščajoča posplošeno Riemannovo integrabilna funkcija potem je d.R.i. in je njen direktni Riemannov integral enak posplošenemu $\int_0^\infty h(u)\,du$.
	\item Če velja $0 \leq h \leq H$, kjer je $H$ d.R.i., $h$ pa merljiva in zvezna skoraj povsod, potem je $h$ d.R.i.
\end{itemize}}
\5

\izrek[Smithov ključni prenovitveni izrek]{Naj bo $h: [0,\infty) \ra \R$ d.R.i. funkcija in $F$ porazdelitvena funkcija nearitmetične nenegativne slučajne spremenljivke $T$. Potem za edino merljivo, lokalno omejeno funkcijo $g$, ki reši enačbp $g = h + g*F$ velja
$$\lim_{t\ra\infty} g(t) ~=~ \frac{1}{\mu} \int_0^\infty h(u)\,du,$$
kjer je $\mu = \int_{[0,\infty)} x\,dF(x) = \E[T] \in (0,\infty]$.}
\5

\trditev[Asimptotična porazdelitev starosti in presežka]{Naj bo $\nt$ prenovitveni proces in $A_t, E_t$ starost in presežek s porazdelitvenima funkcijama $a_x(t), e_x(t)$. Smithov izrek nam pove
\begin{align*}
\lim_{t\ra\infty} \P\set{A_t \leq x} ~&=~ \lim_{t\ra\infty} a_x(t) \\
&=~ \frac{\int_0^\infty \oklepaj{1-F(u)}\1_{[0,x]}(u)\,du}{\mu} \\
&=~ \frac{\int_0^x \oklepaj{1-F(u)}\,du}{\mu}
\end{align*}
in 
\begin{align*}
\lim_{t\ra\infty} \P\set{E_t \leq x} ~&=~ \lim_{t\ra\infty} e_x(t) \\
&=~ \frac{\int_0^\infty \oklepaj{1-F(u)}\,du - \int_x^\infty \oklepaj{1-F(u)}\,du}{\mu} \\
&=~ \frac{\int_0^x \oklepaj{1-F(u)}\,du}{\mu}.
\end{align*}
Za $\mu < \infty$ torej v obeh primerih dobimo, da je asimptotična porazdelitev enaka porazdelitvi integriranega repa.}
\5

\definicija[Prenovitev z defektom]{Naj bo $(T_i)_{i \geq 1}$ zaporedje neodvisnih nenegativnih, $F$-porazdeljenih slučajnih spremenljivk. Pravimo, da so te defektne, če velja
$$\P\set{T=\infty} = 1 - \P\set{T<\infty} ~=~ 1 - F(\infty) ~>~ 0.$$
Da se izognemo trivialnosti, lahko predpostavimo še $F(\infty) > 0$. Za porazdelitev $N_\infty = \lim_{t\ra\infty} N_t$ velja
\begin{align*}
\P\set{N_\infty = k} ~&=~ \P\set{S_k < \infty, S_{k+1} = \infty} \\
&=~ \P\set{T_1<\infty,\ldots,T_k<\infty,T_{k+1}=\infty} \\
&=~ F(\infty)^k\oklepaj{1-F(\infty)},
\end{align*}
torej $N_\infty \overset{(d)}{=} \geom\oklepaj{1-F(\infty)}-1$, torej je neizrojena. Velja tudi
$$\lim_{t\ra\infty} \E[N_t] ~=~ \lim_{t\ra\infty} M(t) ~=~ \frac{F(\infty)}{1-F(\infty)}.$$}
\5

\komentar{V praksi imamo najpogosteje prenovitveno enačbo oblike
$$g ~=~ h + \gamma g * F,$$
kjer je $F$ porazdelitvena funkcija neizrojene nenegativne slučajne spremenljivke ($F(\infty) = 1$) in $\gamma \in (0,1)$.} 
\5

\izrek[Obstoj in enoličnost rešitev]{Naj bo $h: [0,\infty) \ra \R$ merljiva in lokalno omejena, $h(x) = 0$ na $(-\infty,0)$. Potem obstaja natanko ena merljiva in lokalno omejena funkcija $g$, $g(x) = 0$ na $(-\infty,0)$, ki reši prenovitveno enačbo z defektom $g = h + g*F$. Dana je z
$$g ~=~ h + h*M$$
oz.
$$g(t) ~=~ h(t) + \int_{[0,\infty]} h(t-s)\,dM(s), \quad t \geq 0.$$}
\5

\trditev[Asimptotika rešitev]{Naj bo $h: [0,\infty) \ra \R$ merljiva in lokalno omejena, $h(x) = 0$ na $(-\infty,0)$ in naj obstaja $h(\infty) = \lim_{t\ra\infty} h(t)$. Če je $h(\infty) \in \overline{R}$ (iz česar sledi omejenost $h$), potem za edino lokalno omejeno merljivo rešitev $(h,F)$-prenovitvene enačbe z defektom $g$ velja
$$\lim_{t\ra\infty} g(t) ~=~ \frac{h(\infty)}{1-F(\infty)}.$$}
\5

% #################################################################################################

\end{document}